\section{Simulation}
  \subsection{Introduction}
    In the previous section, we analyzed and proved the convergence bounds on
    the number of rounds in which Steemit retrieves optimal results in the
    curation of content. We measure the quality of curation of the platform with
    our $t$-convergence metric, which identifies if the t-top posts in
    the final lists of posts are equal to the t-posts list on the ideal quality
    order. The outcomes obtained in our theoretical analysis of the platform
    have motivated us to better understand the implications of these results in
    practice. In the current section, we create a simulation framework to
    replicate the execution of Steemit's post-voting system, based on the
    protocols of section(x) and Appendix.

    In particular we will consider two scenarios for our simulation: when all
    users are honest (i.e. vote according to the likability distribution) as in
    section X and when users behave selfishly in order to improve the final
    position of their posts. With the first scenario, we aim to understand in
    more depth the practical implementation of the bounds obtained in the
    previous section. We study how the curation quality of the system varies
    with the number of voting rounds to give a better grasp on the curation
    properties a post-voting system can achieve when the number of available
    rounds is tight. The second case tries to measure how resilient is the
    curation quality of Steemit to malicious agents. We study which are the
    effects of a subset of users of the platform not following the likability
    distribution of the posts to cast votes and voting to maximize the
    popularity of their own posts.

    \subsubsection*{Modeling greedy behavior.}
      Steem allows its users to self-vote. Self-voting occurs when a user
      upvotes their own content. While this cannot be regarded as an intrinsic
      flaw in the protocol, if self-voting abusively is profitable, there are no
      incentives for rational players to follow the curation system defined in
      our model by the Likability distribution.

      In Steem, users cannot vote more than once for the same piece of content,
      so greedy users can create dummy posts and constantly vote for them. In
      our model of section X, we establish the condition that users can create
      only one post at the beginning of the voting execution (see algo..).
      Therefore, to replicate the effects of abusively self-voting, we introduce
      \textit{hub-and-spoke} voting rings, where a greedy user create a defined
      number sybil accounts which will vote for the post created by the selfish
      player.

  \subsection{Methodology}
    To measure the curation quality we employ the $t$-convergence used in
    section X. However, our analysis now aims to capture how this metric varies
    depending on the system parameters not in the calculation of a bound for the
    curation success. The evolution of the $t$-convergence on the list of
    posts for different number of rounds will provide a better intuition of
    these research questions. To complement the convergence analysis for the
    top-t posts of the list, we use two statistical coefficients, Kendall's Tau
    and Spearman's Rho, to measure the correlation of the curated list and the
    ideal list of posts based on the likability distribution.

    \subsubsection*{Rank correlation coefficients.}
      To quantify the similarity between two ordered list of posts we employ
      Spearman's Rho and Kendall's Tau, two of the most popular rank correlation
      coefficients~\cite{kendall1955rank}.  When measuring rank correlation,
      these coefficients will measure the statistical significance between two
      lists, retrieving a value $\rho,\tau \in [-1 , 1]$. When the ranking of
      two lists is completely unrelated to each other, the rank correlation
      coefficient will be $0$. If there exists rank correlation, the
      coefficients oscillate between $1$ and $-1$, meaning absolute correlation
      (i.e. the ranks of the lists are equal) and absolute inverse correlation
      (i.e. the lists are ranked in reversed order).\\

    In addition to $t$-convergence and the rank correlation coefficients
    used in the ``all-honest'' scenario, we include a \textit{selfish-rewards}
    metric to quantify how successful for the selfish players was deviating from
    the protocol. In particular, we compare the position of the post created by
    a selfish (or greedy, dunno what's better) user with the ranking of the post
    following the ideal order of qualities. In this way, we are able to measure
    how advantageous is for users to behave greedily and how this behavior
    affects the overall quality of curation of the platform (i.e. using the rest
    of curation quality metrics previously defined).

  \subsection{Results}

    \subsubsection*{Scenario A: All users are honest}
      The simulation scenario in which all users follow the established protocol
      is an extension of the $t$-convergence bounds obtained in the previous
      section and permits us study the curation problem from a more practical
      perspective.

    \subsubsection{Scenario B: Selfish users}
      The inclusion of selfish actors in our simulation of Steemit permits us
      measuring the system's resilience to deviations of the expected curation
      protocol (i.e. all users voting according to their likability). As
      previously mentioned, we constitute voting rings in which one user, the
      ring leader, creates a number of sybil accounts which will vote firstly
      for the post of the ring leader. In order to understand how the presence
      of voting rings affect the quality of curation of the platform, we run our
      simulations varying key parameters such as the ring size or the ration
      between greedy participants and the total number of users. From our
      analysis in Scenario A, we are able to run the simulation in the most
      relevant curation conditions, such as bounding number of rounds when the
      $t$-convergence is reached.
