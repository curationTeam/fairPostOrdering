\section{Related Work}
  Several research efforts have aimed to model the mechanics and incentives for users in crowdsourced content curation systems. Motivated by the widespread adoption of crowdsourced aggregation sites such as Reddit (cite) or Digg, Askalidis et al (cite) formalized the crowdsourced curation of User-generated content (UGC). Along these lines, Ghosh et al. analyzed the incentives and game-theoretic implications of these systems in a series of research publications (cite, cite, cite). We recognize the value of these past efforts and we adopt the main components of these models such as the quality of the articles and the user's attention span. However, our approach is fundamentally different as we start from the analysis of a living platform like Steemit, to further provide a general framework for the analysis of decentralized blogging platforms based on user voting systems. Moreover, while past research efforts have analyzed crowdsourced curation systems from an analytic or game-theoretic angle, we formalize the mechanics of post-voting systems following the cryptography literature (cite) of protocol design against incentivized adversaries.\\
  
  The incentivized post-curation system of Steemit belongs to the recently originated paradigm of blockchain governance. The governance of online communities such as Wikipedia has been thoroughly studied in previous academic work(cite, cite). However, the governance processes in blockchain systems, where the voters are at the same time equity-holders have still many open research questions. Buterin (cite) have criticised the effectiveness of coin-holder voting systems present in decentralized platforms as DAOs or in different blockchain protocols such as EOS(cite) or Tezos(cite). Our analysis of Steemit's post-voting system aims to provide a better framework for the better design in future decentralized curation platforms.
  
  
