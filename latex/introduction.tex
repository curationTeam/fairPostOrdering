\section{Introduction}
  The modern Internet contains an immense amount of data; a single user can only
  consume a tiny fraction in a reasonable amount of time. Therefore, any widely
  used platform that hosts user-generated content (UGC) must employ a content
  curation mechanism. Content curation can be understood as the set of
  mechanisms which rank, aggregate and filter relevant information. In recent
  years, popular news aggregation sites like
  Reddit\footnote{\url{https://www.reddit.com/} Accessed: 2019-01-02} or Hacker
  News\footnote{\url{https://news.ycombinator.com/} Accessed: 2019-01-02} have
  established crowdsourced curation as the primary way to filter content for
  their users. Crowdsourced content curation, as opposed to more traditional
  techniques such as expert- or algorithmic-based curation, orders and filters
  content based on the ratings and feedback of the users themselves, obviating
  the need for a central moderator by leveraging the ``wisdom of the
  crowd''~\cite{askalidis2013theoretical}.

  The decentralized nature of crowdsourced curation makes it a suitable solution
  for ranking user-generated content in blockchain-based content hosting
  systems. The aggregation and filtering of user-generated content emerges as a
  particularly challenging problem in permissionless blockchains, as any
  solution that requires a concrete moderator implies that there exists a
  privileged party, which is incompatible with a permissionless blockchain.
  Moreover, public blockchains are easy targets for Sybil attacks, as any user
  can create new accounts at any time for a marginal cost. Therefore, on-chain
  mechanisms to resist the effect of Sybil users are necessary for a healthy and
  well-functioning platform; traditional counter-Sybil
  mechanisms~\cite{levine2006survey} are much harder to apply in the case of
  blockchains due to the decentralized nature of the latter. The functions
  performed by moderators in traditional content platforms need to be replaced
  by incentive mechanisms that ensure self-regulation. Having the impact of a
  vote depend on the number of coins the voter holds is an intuitively appealing
  strategy to achieve a proper alignment of incentives for users in
  decentralized content platforms; specifically, it can render Sybil attacks
  impossible.

  However, the correct design of such systems is still an unsolved problem.
  Blockchains have created a new economic paradigm where users are at the same
  time equity holders in the system, and leveraging this property in a robust
  manner constitutes an interesting challenge. A variety of projects have
  designed decentralized content curation systems~\cite{synereo,steemit,tcr}.
  Nevertheless, a deep understanding of the properties of such systems is still
  lacking. Among them, Steemit has a long track record, having been in operation
  since 2016 and attaining a user base of more than 1.08
  M\footnote{\url{https://steemdb.com/accounts} Accessed: 2019-01-02} registered
  accounts\footnote{The number of accounts should not be understood as the
  number of active users, as one user can create multiple accounts.}. Steemit is
  a social media platform which lets users earn money (in the form of the STEEM
  cryptocurrency) by both creating and curating content in the network. Steemit
  is the front-end of the social network, a graphical web interface which allows
  users to see the content of the platform. On the other hand, all the back-end
  information is stored on a distributed ledger, the Steem blockchain. Steem can
  be understood as an ``app-chain'', a blockchain with a specific application
  purpose: serving as a distributed database for social media
  applications~\cite{steemit}.

  \noindent  \textbf{Our Contributions.} In this work we study the foundations
  of decentralized content curation from a computational perspective. We develop
  an abstract model of a post-voting system which aims to sort the posts created
  by users in a distributed and crowdsourced manner. Our model is constituted by
  a functionality which executes a protocol performed by $N$ players. The model
  includes an honest participant behaviour while it allows deviations to be
  modeled for a subset of the participants. The $N$ players contribute votes in
  a round-based curation process. The impact of each vote depends on the number
  of coins held by the player. The posts are arranged in a list, sorted by the
  value of votes received, resembling the front-page model of Reddit or Hacker
  News. In the model, players vote according to their subjective opinion on the
  quality of the posts and have a limited attention span.

  Following previous related
  work~\cite{ghosh2011incentivizing,askalidis2013theoretical}, we represent each
  player's opinion on each post (i.e. likability) with a numerical value $\like
  \in [ 0,1 ]$. The objective quality of a post is calculated as the simple
  summation of all players' likabilities for the post in question. To measure
  the effectiveness of a post-voting system, we introduce the property of
  \textit{convergence} under honesty which is parameterised by a number of
  values including a metric $t$, that demands the first $t$ articles to be
  ordered according to the objective quality of the posts at the end of the
  execution assuming all participants signal honestly to the system their
  personal preferences. Armed with our post-voting system abstraction, we
  proceed to particularize it to model Steemit and provide the following
  results.

  \begin{itemize}
    \item[i)] We characterise the conditions under which the Steemit algorithm
    converges under honesty. Our results highlight some fundamental limitations
    of the actual Steemit parameterization. Specifically, for curated lists of
    length bigger than 70 the algorithm may {\em not achieve even
    1-convergence}.
    \item[ii)] We validate our results with a simulation testing different
    metrics based on correlation that have been proposed in previous
    works~\cite{kendall1955rank,spearman1904proof} and relating them to our
    notion of convergence.
    \item[iii)] We demonstrate that ``selfish'' deviation from honest behavior
    results to substantial gains in terms of boosting the ranking of specific
    posts in the resulting list of the post-voting system.
  \end{itemize}
%   Our analysis of the post-voting system of Steemit aims to provide a useful
%   framework to aid the design of future decentralized curation platforms.

%  \begin{itemize}
%  \item Original treatment following cryptography and simulation-based
%  techniques to model crowdsourced content curation, in opposition to previous
%  academic work which have leaned more towards an analytical and game-theoretic
%  approach.

%  \item Taking in account of subjective likabilities, the effect of rounds and
%  an agnostic likability distribution treatment.

%  \item Measurement of the influence of coin-holding (wealth-distribution) in
%  the effectivity of post sorting. New paradigm present in blockchain based
%  systems (main difference with Reddit-like platforms.)

%  \item Proof that Steem does not t-converge with the parameters currently used
%  in their implementation. Insights to improve the curation quality of the
%  trending section of steem.

%  \item (Maybe)Impact of the curation quality when greedy players are present in
%  the system.
%  \end{itemize}
