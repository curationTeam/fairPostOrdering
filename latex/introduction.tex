\section{Introduction}
    
  The huge amount of available information on the Internet makes essential the aggregation and curation of content to fitting the attention span of online users. Content curation can be understood as the set of mechanisms which rank, aggregate and filter information relevant to one or more users. In recent years, popular news aggregation sites like Reddit or Hacker News have established crowdsourced curation as the primary way to filter content for their users. Crowdsourced content curation, as opposed to more traditional techniques such as experts or algorithmic-based curation, orders and filters content based on the ratings and feedback of the users themselves. As such, crowdsourced content curation provides a mechanism to aggregate content without relying on a central party in the form of a moderator.\\
  
  The distributed nature of crowdsourced curation makes it a suitable solution for sorting and ranking problems in blockchain-based systems. We identify decentralized content curation as a subset of problems within the broader spectrum of blockchain governance. Governance in blockchains can be understood as the set of decision-making processes established to collaboratively reach agreements and make decisions among the token-holders of the network. In these systems, the influence of each participant in the voting process is weighted with the number of tokens she holds or has vested in the network. We divide the use cases of blockchain governance in two main categories: agreement or decision-making processes and ranking or sorting problems. A relevant example which falls in the first category is the agreement among miners in the minimum acceptable gas price in the Ethereum network. On the other hand, the election of miners or validators in EOS and Steem blockchain illustrate the second category. EOS and Steem blockchains reach consensus via the DPoS algorithm, in which token-holders vote to decide a subset of nodes which will become the block validators of the network, based on the number of votes each one has received. In this paper, we will solely focus on the properties of the second category of blockchain governance processes.\\
  
  
  In this paper, we develop a theoretical model of a post-voting system which ranks and sorts the posts in a distributed and crowdsourced manner. Our model is constituted by N players which participate in a round-based curation process and the value of each vote will be dependent on the number of tokens held by the player in the network. The posts will be arranged in a list, sorted by the value of votes received, resembling to the front-page model of Reddit or Hacker News. In the model, players vote according to the quality of the posts and have a limited attention span $AS$. Following previous academic work, we model the article's quality with a numerical value $n \in \lbrace 0,1 \rbrace$. Moreover, we refine the properties of the system by including subjective likeabilities among different users and remaining agnostic in terms of the distribution of these likabilities. The objective quality of a post will be calculated as the simple summation of the likability of the post for all the players in the system. To measure the properties of such a content-voting system, we say that the system \textit{t-converges} if the t first articles after all voting rounds have been executed are ordered according to the objective quality of the posts.
  After defining and formalizing a general post-voting system, we particularize it to model Steemit, a social media platform á la Reddit based on the Steem blockchain. We develop a computer-based simulation to help us parametrize Steemit according to our post-voting model and then we prove the conditions under which Steemit t-converges (i.e. is successful in the content curation process). TO BE EXTENDED.\\
  
  \textbf{Our Contributions.}  
  
  \begin{itemize}
  
  \item Original treatment following cryptography and simulation-based techniques to model crowdsourced content curation, in opposition to previous academic work which have leaned more towards an analytical and game-theoretic approach.
  
  \item Taking in account of subjective likeabilities, the effect of rounds and an agnostic likeability distribution treatment.
  
  \item Measurement of the influence of token-holding (wealth-distribution) in the effectivity of post sorting. New paradigm present in blockchain based systems (main difference with Reddit-like platforms.)
   
  \item Proof that Steem does not t-converge with the parameters currently used in their implementation. Insights to improve the curation quality of the trending section of steem.
   
  \item (Maybe)Impact of the curation quality when greedy players are present in the system.
   
  \end{itemize}
    

  
  
  